\documentclass[14pt, a4paper]{extarticle}
\usepackage{xltxtra}

\usepackage{polyglossia}
\setmainlanguage{russian}
\setotherlanguage{english}
\setkeys{russian}{babelshorthands=true}

\setmainfont{Times New Roman}
\setromanfont{Times New Roman} 

\newfontfamily{\cyrillicfont}{Times New Roman} 
\newfontfamily{\cyrillicfontrm}{Times New Roman}

\usepackage[T2A]{fontenc}   % кодировка
\usepackage[utf8]{inputenc}
\usepackage{cmap}     % поиск в PDF
\usepackage{amsmath}                         %Подключает математический пакет
\usepackage{amsfonts}                          %Математические шрифты
\usepackage{amsmath,amsthm,amssymb} 
\usepackage{color}     
\usepackage{sectsty} 
\allsectionsfont{\centering}
\usepackage{geometry}
\geometry{a4paper, top=2.5 cm, left=1.5cm, right=1.5cm, bottom=1.5cm}
\usepackage{indentfirst}
% \usepackage{anyfontsize}
\usepackage{setspace}
\onehalfspacing
\usepackage{fontspec}
\usepackage{enumitem}
\usepackage{hyperref}
\usepackage{mathtools}
\usepackage{setspace}
\usepackage{tikz}
\usepackage{polyglossia}

\title{Анализ локализации "узких мест" в мировом производстве компонент компьютерной техники}
\author{Евланов Максим}
\date{Декабрь 2023}
\begin{document}
\maketitle
\newpage 

\tableofcontents{\vspace{3cm}}
\newpage 


\section{Введение}

Тематика анализа локализации "узких мест"\ в мировом производстве компонент компьютерной техники приобретает всю большую актуальность в связи с ростом потребности в вычислительных мощностях мировой экономикой и недавними мировыми кризисами, вызвавшие перебои в мировых цепочках поставок. Для раскрытия данной темы необходимо требует разобраться в специфике вопроса как и производства современной микроэлектроники, так и в особенностях мировой торговли. Литература на тему мирового производства компонент компьютерной техники не столь обширна в связи с тем, что данная проблема появилась сравнительно недавно. Тема компонент компьютерной техники довольно обширна, но почти все компоненты завязаны на чипах - важнейшей части микроэлектронной промышленности, используя технологию полупроводников и их производства. В данном обзоре производство компонент компьютерной техники приравнены в рынку микроэлектроники, полупроводников и чипов, так как в большинстве статьях они рассмотрены в тесной связи между собой.

Целью данной работы является комплексный и критический обзор литературы на тему "Анализ локализации "узких мест"\ в мировом производстве компонент компьютерной техники"\ , включающий в себя изучение имеющейся литературы и её анализ". 

\newpage 
\section{Анализ содержания статей}

На сегодняшний день данной темой интересуются разные авторы. Так, к примеру, в статье \begin{english} ``Strengthening the Global Semiconductor Supply Chain in an Uncertain Era"\end{english} (2021) авторами A.Varas, R.Varadarajan, J.Goodrich, F.Yinug выделяется то, что специализация отдельных стран помогала внедрению и созданию инноваций, а также поддержке низкие цены на рынке микроэлектроники. Однако авторы подчёркивают, что эта же географическая распределённость является глобальной причиной множества природных и политических рисков в мировом производстве чипов, и предлагают стимулировать государству внутреннее производство микроэлектроники.

Данному направлению посвящена также свежая статья "Импортозамещение на рынке компьютерной техники специального назначения: возможные пути реализации"\   (2023), где авторами В.А.Дуболазов и В.С.Силинский отмечается, что предприятия, производящие компьютерную технику специального назначения (КТСН), востребованную в стратегических отраслях страны, играют огромнейшую роль в экономики государства, а также подчёркивается, что в условиях экономических санкций необходимо как развивать и внедрять импортозамещение, так и создавать высокотехнологичное производство. Также авторами выделяется, что главными составляющими являются, как и наращивание внутренней составляющей в электроники, так и развитие общей стратегии по импортозамещению. В научной работе отмечается высокая зависимость отечественной высокотехнологичной отрасли от иностранных производителей и важность внедрения процесса импортозамещения. Главная цель статьи - создание набора решения для решения вопросов внутреннего производства, а главная методика - эмпирический сбор информации.

Иные заключения содержатся в статье \begin{english} "Securing Semiconductor Supply Chains: An Affirmative Agenda for International Cooperation"\end{english} (2022). В данной статье авторы W.A.Reinsch, E.Benson, A.Arasasingham выделяют на примере США особенное влияние полупроводников на технологические и лидирующие отрасли экономики и влияние полупроводниковых кризисов последних лет на всю экономику. Исследователи приходят к выводам, что внутреннее производство чипов связано со множеством ограничений и в краткосрочной перспективе невозможно, однако в долгосрочной перспективе перенос производства в другие страны постепенно будет повышать издержки и приведёт к уменьшению конкурентоспособности США. Авторы подчёркивают необходимость США укреплять цепочки поставок среди своих союзников по производству чипов, а также ослабить экспортные ограничения для привлечения инвестиций и лучшей кооперации.

В свежей статье "Three Reasons Why CHIPS-plus is a Big Win for US National Security"\ (2022) автором T.Klein отмечается важность принятия нового закона в США, который выделяет новые средства для изучения отрасли полупроводников, а также субсидии для внутреннего производства чипов. Автор рассматривает данные государственные инвестиции в отрасль как инвестицию в будущее американской национальной безопасности и инноваций, а также меру по стимулированию конкурентоспособности США на мировом рынке. Увеличение внутреннего производства поможет США не зависеть от мировой геополитической неопределённости.

Конкуренции стран в производстве полупроводников и местам напряжения посвящена статья "Chapter 3. Winning interdependence: semiconductor and CRM rivalry in a de-globalising world" (2022), где авторами исследования \begin{english} J.Teer, M.Bertolini, J.A.Ritoe, S.Heyster, T.Sweijs, R.Wijk, M.Rademaker, M.Vlaskamp, I.Patrahau, J.Thompson, S.Kim, R.Minicozzi, A.Meszaros, G.Cisco, M.Gorecki\end{english} анализируются такие вопросы, как равновесие в производстве полупроводников и то, какие риски несёт возрастающая мировая напряжённость в мировых цепочках поставок. Авторами отмечается, что монополия, сформированная странами, сотрудничающих в сфере производства полупроводников, может использоваться как фактор давления на их политических конкурентов и активно используется. Также данная коалиция стремится укрепить своё доминирование путём ограничения экспорта полупроводников и технологий для их производства в страны-конкуренты и давления на страны, предоставляющие высокотехнологичное оборудование, как, например, давление на абсолютного монополиста на рынке высоко-технологического литографического оборудование Нидерландская компания ASML.

Акцент на глобальном взаимодействии стран в производстве полупроводников сделан в работе "Expanding the Depth and Breadth of the US-Taiwan Technological Partnership via the Semiconductor Ecosystem" (2022) автором A.Wylegala, исследователь разобрал случай сотрудничества США и Тайваня в сфере микроэлектроники. Автор отмечает, что долгая история близкого сотрудничества стран принесла обеим такие плоды, которые не были бы возможны без сотрудничества. Также исследователь подчёркивает, что США и Тайваню и далее следует укреплять связи в производстве и сотрудничестве и рассматривать друг друга как ключевых стратегических партнёров, и это сотрудничество и развитие является перспективой в цифровой экономике.

В недавней статье "Securing the Microelectronics Supply Chain: Four Policy Issues for the U.S. Department of Defense to Consider" (2022) исследователями J.Mondschein, J.W.Welburn, D.Gonzales  рассматривается 4 ключевых вопросов американской политики по отношению к мировым цепочкам производства микроэлектроники. Авторы выделили четыре ключевых вопроса, определяющих политику США, связанные с цепочками поставок микроэлектроники и необходимостью создания стратегии управления рисками в этой области. Они отмечают, что глобальный дефицит полупроводников подчеркнул уязвимость цепочек поставок, и что усиление внутреннего производства и применение стратегий управления рисками могут смягчить некоторые из этих рисков. Авторы также обсуждают необходимость скоординированных усилий по управлению рисками в цепочках поставок микроэлектроники и оптимальное сочетание политических мер для продвижения экосистемы технологий микроэлектроники, соответствующей стратегическим целям США.

Авторы S.Shivakumar, C.Wessner, T.Howell статьи "The Pillars Necessary for a Strong Domestic Semiconductor Industry" (2022) рассматривают вопрос переноса производства полупроводников внутрь страны. Они описывают с какими потенциальными проблемами столкнётся США при развитии внутреннего производства, такими, как крупные капиталовложения, значительные инвестиции в инновации и развитие рабочей силы. Исследователи делают вывод, что необходимо реагировать на современные потребности, анализируя опыт компании Sematech. Успех Sematech был обусловлен рядом факторов, включая срочную необходимость возвращения индустрии чипов США в состояние конкурентоспособности с Японией, четкую цель, сохранение фокуса исследований, долгосрочной поддержки и финансирования, использование существующих активов, участие заинтересованных сторон и получение опыта на ошибках других организаций, в том числе в прошлом

Подробное описание полупроводниковой промышленности и то, как особенности технологического процесса влияют на мировые цепочки производства чипов, представлена в статье "Review of the Semiconductor Industry and Technology Roadmap"\ (2002) авторами S.Kumar and N.Krenner. Исследователи отмечают, что полупроводниковая промышленность находится в состоянии постоянного снижения цен, что является критически важным для нашего развития и прогресса в эпоху знаний. Авторы подмечают, что данная сфера является очень конкурентоспособной и требует непрерывного технологического прогресса для увеличения производительности и снижения затрат, чтобы оставаться на рынке, а также важно понимать основные аспекты этой отрасли, включая историю полупроводников, типы продукции и ее использование, процесс изготовления полупроводниковых чипов, общий обзор отрасли и тенденции. Технологическая дорожная карта является одним из основных инструментов, используемых в отрасли для управления разработкой продуктов и технологий, а изучение первоначальной и текущей дорожной карты полупроводниковых технологий поможет лучше понять эту отрасль.

В книге "Global Value Chains: Linking Local Producers from Developing Countries to International Markets" (2012) авторами P.Goes, M.P.Dijk одна из глав \begin{english} "Global Competition in the Semiconductor Industry: A Comparative Study of Malaysian and Chinese Semiconductor Value Chains" \end{english} посвящена созданию высоко конкурентных цепочек производства в юго-восточной Азии. Авторы отмечают, что азиатские страны привлекательны для размещения предприятий по сборке электроники из-за низких производственных затрат, и полупроводники играют важную роль в современных электронных системах, и непрерывные инновации являются ключевым аспектом полупроводниковой промышленности. Исследователи проводят сравнение полупроводниковой промышленности Китая и Малайзии с целью выяснить, насколько устойчивы показатели экономического роста и существует ли движение к деятельности с более высокой добавленной стоимостью, и отмечают три основных вопроса, включая конкуренцию между полупроводниковыми отраслями обеих стран, условия для развития экономики знаний и другие конкурентные условия, а также перспективы каждой из стран в глобальной цепочке создания стоимости полупроводников.

V.Kannan, J.Feldgoise  в статье "Goals and Limitations of the CHIPS Act" (2022) анализируют такие вопросы, как "CHIPS Act"\ - законопроект США, что предоставляет всестороннюю поддержку индустрии полупроводников. Авторы выделяют несколько рекомендаций для правительства США по улучшению безопасности и устойчивости цепочки поставок полупроводников и подчёркивают потенциал полупроводниковой промышленности для повышения международной конкурентоспособности и внутренней занятости. Исследователи отмечают следующие рекомендации: совместная работа правительства и промышленности, консультации с ведущими учеными для изучения того, какие экономические политики и инициативы помогут использовать внутреннюю рабочую силу, а также финансирование НИОКР и поддержка инициатив, готовящие американские компании новым вызовам полупроводниковых технологий.

Подробный анализ истории и опыта развития технологично промышленности Тайваня был изложен E.A.Feigenbaum в статье \begin{english} "Historical Context of Taiwan’s Technological Success"\end{english} (2020). Автор отмечает, что Taiwan Startup Stadium помогает местным финтех-стартапам связаться с региональными и глобальными программами финтех-акселераторов, а также другие тайваньские акселераторы, такие как AppWorks, сфокусировались на новых технологиях, таких как блокчейн, и привлекли стартапы из Юго-Восточной Азии, которые ищут инвестиции в тайваньскую индустрию венчурного капитала и корпоративное партнерство на Тайване. Главная цель статьи - выявить препятствия, которые мешают развитию инвестиций в будущем в Тайване, предложить конкретные идеи правительству, промышленности и игрокам рынков капитала, чтобы помочь смягчить и преодолеть некоторые из этих препятствий, и изучить, как перспективное партнерство между игроками Тайваня и США может способствовать реализации этих идей.

В статье "The Strategic Importance of Legacy Chips" (2023) авторы исследования S.Shivakumar, C.Wessner, T.Howell рассматривают вопрос нехватки уже устаревших чипов. Исследователи отмечают, что в 2020 году дефицит чипов показал, что самые передовые полупроводники больше не производятся в США, что создаёт стратегическую уязвимость. Авторы отмечают, что этот дефицит в основном был вызван недостаточной доступностью устаревших чипов, которые ещё производятся американскими компаниями, но в недостаточных количествах. Он повлиял на различные отрасли, включая автомобилестроение, а также на устройства, использующие различные полупроводниковые технологии. Это привело к серьезным потрясениям в экономике США и подчеркнуло стратегическое значение устаревших чипов.

В книге "When the Chips Are Down: Gaming the Global Semiconductor Competition"\ (2022), что содержат две ценные научные статьи \begin{english}  "Securing Semiconductors: Recommendations for the United States"\end{english}  и "The Chips Are Down: The Game", авторы B.Wasser, M.Rasser, H.Kelley рассматривают вопрос противостояния обеспеченного ресурсами Китая и США в сфере микроэлектроники. Исследователи подчёркивают, что ключевым фактором соперничества является технологии, связанные с полупроводниками. Авторы выделяют такие рекомендации для США, как сосредоточиться на взаимовыгодном совместном сотрудничестве с Тайванем, укрепить кооперацию ведомств для противостоянию давления Китая на Тайвань и обеспечению безопасности цепочки производства полупроводников. Исследователи делают выводы о стратегической важности полупроводников и Китай не станет ждать помощи от США. 

\newpage 
\section{Критическое сравнение полученных в них результатов. Анализ и оценка недостатков и преимуществ, полученных в статьях результатов и сделанных в них допущений}
В данном разделе будет проведён критический анализ статей и их сравнение, а также выделение сильных и слабых сторон представленных статей. 

В представленных научных статьях рассматриваются множество вопросов в современном производстве полупроводников. Изучались такие вопросы как истории развития полупроводниковой промышленности, текущей ситуации, безопасности существующих цепочек производства, возможность их переноса, вопросы сотрудничества, опыт развития полупроводниковой отрасли в разных странах, а также возможность развития "внутренней"\ полупроводниковой промышленности.

В статье \begin{english} "Strengthening the Global Semiconductor Supply Chain in an Uncertain Era"\end{english} обсуждается текущее положение производства микроэлектроники, подчёркивается, что специализация стран стимулировала инновации в столь сложной отрасли, и выделяются такие факторы риска, как географическая распределённость, а также приводится довольно обширная статистика распределения производства и технологий по разным странам, проводится комплексный анализ данного рынка, что является большим и важным фактором ценности статьи. В статье \begin{english} "Review of the Semiconductor Industry and Technology Roadmap"\end{english} авторы подробно описывают техническую карту полупроводниковой промышленности, опираясь на статистику и историю индустрии, делают акцент на иную особенность создания чипов, связанную с постоянной потребностью инноваций и снижения стоимости производства, требующих больших капиталовложений, что также является риском и проблемой, и это сильная сторона данной статьи.

В статье \begin{english} "Chapter 3. Winning interdependence: semiconductor and CRM rivalry in a de-globalising world"\end{english} авторы делают подробный анализ рисков и выделяют мировую напряжённость как одним из самых значимых факторов, а также приводится необычный пример того, как главные игроки могут давить на другие страны и компании, в том числе на друг друга, что явно является преимуществом проделанного исследования, а также дополнением уже имеющихся выводов. В статье же \begin{english} "Securing the Microelectronics Supply Chain: Four Policy Issues for the U.S. Department of Defense to Consider"\end{english} авторы, помимо выделения схожих рисков, как в прошлых статьях, предлагают некоторые рекомендации (на примере США), акцентируя внимание на укреплении уже имеющегося сотрудничества между странами-партнёрами, решительной политической консолидации, а также на развитии внутреннего рынка. Однако, в сравнении с прошлыми статьями, статья предоставляет более скудный анализ, лишь описывая проблемы и необходимые решения с точки зрения военной безопасности США. 

В книге \begin{english} "When the Chips Are Down: Gaming the Global Semiconductor" \end{english} авторы обращают подробнее на проблему, что озвучивалась в других источниках, о том, которая китайская микроэлектронная промышленность во многом уже приспособилась ко всем ограничениям и взяла курс на "самодостаточность"\ , не ожидая помощи от соперников и именно поэтому нужно усилить координацию между США и Тайванем - главными партнёрами в сфере чипостроения. 

Необычная точка зрения предоставлена свежей в статье \begin{english}  "The Strategic Importance of Legacy Chips"\end{english}, где авторы вновь обращаются к проблемам полупроводниковых кризисов, однако делают акцент на менее продвинутых чипах, которые также могут быть использованы в промышленности для более простых нужд, однако их производство во многом не развито. Главная проблема данной статьи - нет обоснования экономической целесообразности инвестиций в расширение производства более старых чипов в эпоху бурного развития передовых чипов.

Авторы статьи \begin{english} "Expanding the Depth and Breadth of the US-Taiwan Technological Partnership via the Semiconductor Ecosystem"\end{english} продвигают идею сотрудничества США и Тайваня, предложенную в более ранних статья, онадко имеет малый объём, слабую анилитику и не представляет особого интереса, во многом может быть названа вторичной. 

История развития индустрии производства чипов подробно освящается в статьях \begin{english}"Historical Context of Taiwan’s Technological Success"\end{english} и \begin{english} "Global Competition in the Semiconductor Industry: A Comparative Study of Malaysian and Chinese Semiconductor Value Chains"\end{english}, где описана как и история развития полупроводниковой отрасли в азиатских странах, так и сравнение данных отраслей и соперничества. Данная информация полезна для полноценной картины рынка чипов, так как показывает условия, в которых страны развивали свою промышленность (например, как на Тайване привлекают инвестиции и развивают инновационные компании), а также те проблемы, с которыми данные страны столкнулись, и их перспективы. В статье про Китай и Малайзию показана довольно хорошая статистика  и её анализ, история развития, а также современные проблемы. Авторы статьи про Тайвань проводят более скромный анализ, но показывают историю развития индустрии. Суммируя данные статьи, они предоставляют важную информацию, которая может быть использована для развития подобной промышленности в иных странах. Или же, например, дать понять, что успех Тайваня и ряда других стран связан с тесным сотрудничеством с США и потребностью переноса производства в страны с более дешевой рабочей  силой и иные тонкости. 

В ряде следующих статей авторы продвигают идею переноса производства внутрь страны и его проблематики.\begin{english} "Three Reasons Why CHIPS-plus is a Big Win for US National
Security"\end{english} авторы обсуждают важность нового закона США по усилению развития внутреннего производства чипов (которое по большей части располагается в азиатских странах)  и инвестиции в инновации. В статье подробно обсуждается важность данного решения, однако нет анализа сложностей, связанные с развитием внутреннего производства. А в статье \begin{english} "Goals and Limitations of the CHIPS Act"\end{english} авторами обсуждается данный закон, однако они концентрируются на тех мерах, которые следует принять  и финансировать, а также говорят о сложностях, с которыми столкнётся США при развитии своего внутреннего производства, проводя довольно хорошую аналитику. 

Продолжает данную тему статья \begin{english}"Securing Semiconductor Supply Chains: An Affirmative Agenda for International Cooperation"\end{english}, где авторы концентрируют внимание на то, что США стратегически необходимо внутреннее производство, но очень сложно в реализации в краткосрочной перспективе, однако перенос производства в будущем даст свои негативные эффекты, отрасль может деградировать в США, понижая конкурентоспособность страны. В статье авторы приводят довольно обширную статистику и аналитику и делают важные выводы, одновременно подчёркивая проблему и кризиса мирового производства, и проблему развития внутренней промышленности. 

Также о тех сложностях, которые стоит учитывать при развитии внутреннего рынка производства полупроводников, пишут авторы статьи \begin{english} "The Pillars Necessary for a Strong Domestic Semiconductor Industry"\end{english}. США и ранее сталкивалась с проблемами падением конкурентоспособности по сравнению со стратегическим партнёром Японией, однако преодолело тот кризис. Исследователи берут в качестве основы историю развития компании Sematech и выделяют ряд факторов, основанных на крупных государственных инвестициях, чёткой цели и принятия опыта прошлого. В самой статье приводится комплексный анализ по разным аспектам. 

Про развитие внутреннего рынка компьютерной техники в России написано в статье "Импортозамещение на рынке
компьютерной техники специального назначения: возможные пути реализации". В данной статье авторы указывают множество проблем российского рынка, связанные с импортозависимостью и изоляцией, делается акцент на необходимость импортозамещения. Однако, в данной статье не приводится методы, которыми оно будет обеспечено, хоть и уточняется, что это сложный процесс. 

\newpage 
\section{Заключение}
Индустрия производства компонент компьютерной техники имеет множество рисков и узких проблем, которые были подчёркнуты исследователями в статьях, представленных выше. Среди главных рисков и "узких мест"\ - возрастающая геополитическая напряженность, техноёмкость, ресурсоёмкость и капиталоёмкость индустрии, отсутствие возможности резкого увеличения масштаба производства, зависимость стран от текущих цепочек поставок, а также зависимость от образованных монополий. Во многих статьях предложены пути решения данных проблем, однако данная отрасль и ситуация вокруг неё слишком сложна для точного прогнозирования. 

\newpage 
\section{Источники}
1) \begin{english} A.Varas, R.Varadarajan, J.Goodrich, F.Yinug (2021). Strengthening the Global Semiconductor Supply Chain in an Uncertain Era. \end{english}

2) В.А.Дуболазов, В.С.Силинский (2023). Импортозамещение на рынке компьютерной техники специального назначения: возможные пути реализации.

3) W.A.Reinsch, E.Benson, A.Arasasingham (2022). Securing Semiconductor Supply Chains: An Affirmative Agenda for International Cooperation.

4) T.Klein (2022). Three Reasons Why CHIPS-plus is a Big Win for US National Security.

5) \begin{english} J.Teer, M.Bertolini, J.A.Ritoe, S.Heyster, T.Sweijs, R.Wijk, M.Rademaker, M.Vlas-\\kamp, I.Patrahau, J.Thompson, S.Kim, R.Minicozzi, A.Meszaros, G.Cisco, M.Gorecki (2022). Chapter 3. Winning interdependence: semiconductor and CRM rivalry in a de-globalising world. \end{english}

6) \begin{english} A.Wylegala (2022). Expanding the Depth and Breadth of the US-Taiwan Technological Partnership via the Semiconductor Ecosystem. \end{english}

7) J.Mondschein, J.W.Welburn, D.Gonzales (2022). Securing the Microelectronics Supply Chain: Four Policy Issues for the U.S. Department of Defense to Consider.

8) S.Shivakumar, C.Wessner, T.Howell (2022). The Pillars Necessary for a Strong Domestic Semiconductor Industry.

9) \begin{english} S.Kumar and N.Krenner (2002). Review of the Semiconductor Industry and Technology Roadmap. \end{english}

10) P.Goes, M.P.Dijk (2012). Global Value Chains: Linking Local Producers from Developing Countries to International Markets.

11) V.Kannan, J.Feldgoise (2022). Goals and Limitations of the CHIPS Act.

12) E.A.Feigenbaum (2020). Historical Context of Taiwan’s Technological Success.

13) S.Shivakumar, C.Wessner, T.Howell (2023). The Strategic Importance of Legacy Chips.

14) B.Wasser, M.Rasser, H.Kelley (2022). When the Chips Are Down: Gaming the Global Semiconductor Competition.

\end{document}